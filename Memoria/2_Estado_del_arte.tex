\section{Estado del arte}
La seguridad vial continúa siendo una prioridad a nivel global. En España, según los datos publicados por la Dirección General de Tráfico (DGT), durante 2024 se produjeron 1040 siniestros mortales, en los que fallecieron 1154 personas y 4634 resultaron heridas graves que requirieron hospitalización \cite{DGT}.
Aunque estas cifras suponen una mejora respecto a años anteriores, reflejan la necesidad de seguir avanzando en sistemas inteligentes de transporte (ITS) que contribuyan a reducir los accidentes y la mortalidad en carretera.

A nivel internacional, los datos son igualmente significativos. En Corea del Sur, durante el año 2022 se registraron 196836 accidentes de tráfico con 2735 fallecimientos, una reducción del 6.2\% respecto al año anterior. Sin embargo, el número de muertes en accidentes que involucraron vehículos de dos ruedas, como motocicletas o patinetes eléctricos, aumentó un 36.8\% \cite{V2X_Corea}. Para reducir estos accidentes y fallecimientos, se están desarrollando CITS (Cooperative Intelligent Transport Systems) y servicios de conducción autónoma que combinan la fusión de sensores y las tecnologías de comunicación de vehículo a todo (V2X).

El artículo añade que los sensores habituales en conducción autónoma (como las cámaras, radares y lidar) presentan limitaciones bajo condiciones meteorológicas adversas, motivo por el cual se pone de manifiesto la relevancia mundial del desarrollo de las tecnologías V2X, orientadas a mejorar la seguridad vial mediante la comunicación cooperativa entre vehículos (V2V) y entre vehículos e infraestructuras (V2I).

En este marco se sitúa el presente trabajo, que estudia y evalúa la comunicación V2V mediante mensajes CAM (Cooperative Awareness Messages) utilizando la plataforma OpenC2X y el simulador de tráfico SUMO, con el objetivo de analizar el comportamiento y las prestaciones de las comunicaciones vehiculares en entornos controlados y reproducibles.

\subsection{Tecnologías V2X y requisitos de servicio}
    \subsubsection{Comunicaciones V2X basadas en ITS-G5}
    El estándar ETSI ITS-G5, constituye la base tecnológica de la comunicación cooperativa vehicular en Europa \cite{Demo_OpenC2X}.

    OpenC2X se ha consolidado como una plataforma de software abierta que implementa la pila ETSI ITS-G5 sobre IEEE 802.11p, con un diseño modular y extensible. OpenC2X consta de módulos altamente independientes, que pueden ampliarse y reutilizarse fácilmente. Esta característica permite que cada componente se ejecute de forma independiente, simplificando el desarrollo y las pruebas.

    OpenC2X implementa los principales servicios definidos por ETSI, incluyendo los mensajes CAM (Cooperative Awareness Message) y DENM (Decentralized Environmental Notification Message), así como el DCC (Decentralized Congestion Control). Este último gestiona la carga del canal inalámbrico ajustando la frecuencia de transmisión según la densidad de tráfico, empleando diferentes estados operativos (relaxed, active o restricted) y el mecanismo EDCA para priorizar los mensajes críticos.

    Los mensajes CAM se generan de forma periódica cuando cambia la posición, la velocidad o el rumbo al menos una vez por segundo. El módulo LDM (Local Dynamic Map) almacena tanto los mensajes enviados como los recibidos, lo que permite monitorear en tiempo real y realizar un análisis posterior.

    En cuanto a la información de detección, la plataforma utiliza el servicio gpsd de Linux para obtener datos de posición, velocidad y tiempo, y admite el uso de trazas GPS previamente registradas.

    \begin{figure}[h]
        \centering
        \includegraphics[width=0.7\textwidth]{images/arquitectura_OpenC2X.PNG}
        \caption{High-level architecture of OpenC2X realizing ETSI ITS-G5 stack.
        The modules communicate with each other via ZeroMQ. Dashed modules are
        not yet implemented in OpenC2X.}
        \label{fig:arquitectura_OpenC2X}
    \end{figure}

    Las pruebas de realizadas con equipos comerciales demuestran interoperabilidad del entorno, recibiendo y enviando mensajes CAM y DENM con un Cohda MK5.

    \subsubsection{Comunicaciones V2X basadas en C-V2X}
    En paralelo a las tecnologías ITS-G5, la industria de las telecomunicaciones ha desarrollado la familia Cellular V2X (C-V2X) \cite{Evolutionary_V2X}, que aprovecha la infraestructura móvil existente y la interfaz PC5 (sidelink) para la comunicación directa entre vehículos. 

    C-V2X combina dos modos de transmisión complementarios: el modo LTE-PC5, que permite la comunicación directa entre vehículos dentro de un área de proximidad sin pasar por la infraestructura, y el modo LTE-Uu, que utiliza la red celular cuando el vehículo se encuentra bajo cobertura.

    Una de las ventajas fundamentales de C-V2X frente a ITS-G5 es su esquema de acceso al medio. Mientras que 802.11p utiliza un mecanismo de contienda CSMA/CA, C-V2X emplea un planificador (scheduler) para asignar recursos de radio de forma controlada. Esto permite reducir colisiones e interferencias, además de ofrecer garantías de calidad de servicio (QoS).

    \subsubsection{Comunicaciones V2X basadas en 5G NR V2X}
    Con la llegada de 5G NR (New Radio) V2X  \cite{V2X_tutorial} se introducen funcionalidades avanzadas diseñadas para soportar los casos de uso más exigentes de la conducción conectada y automatizada.

    NR V2X está diseñado para soportar requisitos de calidad de servicio (QoS) diversos y estrictos, definidos en términos de prioridad, velocidad de transmisión, latencia, fiabilidad, velocidad de datos y alcance de comunicación. Esta arquitectura define flujos QoS asociados a diferentes tipos de comunicación (unicast, groupcast y broadcast), cada uno con sus propios niveles de prioridad y gestión de recursos.

    El modelo QoS de NR V2X se basa en el de 5G, utilizando reglas PC5 QoS y flujos QoS Flows mapeados sobre portadoras de radio laterales (SLRB). Esta estructura permite mantener garantías de servicio incluso fuera de cobertura.

    Los requisitos definidos por 3GPP para aplicaciones críticas, como el platooning, son especialmente exigentes: latencias extremo a extremo de 10 a 25ms, fiabilidad de 90$-$99.99\% y rangos de comunicación de 80$-$350m.
    \begin{figure}[h]
        \centering
        \includegraphics[width=1\textwidth]{images/Requerimientos_3GPP.PNG}
        \caption{Rangos de requisitos para los casos de uso de 3GPP.}
        \label{fig:arquitectura_OpenC2X}
    \end{figure}

    Asimismo, países como Corea del Sur han impulsado proyectos nacionales de despliegue de 5G NR V2X, destacando que esta tecnología admite velocidades ultra altas, retrasos ultra bajos y alta fiabilidad, con objetivos de más de 150 Mbps de capacidad, latencias inferiores a 3 ms y fiabilidad superior al 99.99\%.

    En conjunto, las soluciones NR V2X suponen una evolución sustancial respecto a ITS-G5, al ofrecer mayor capacidad de gestión de la calidad de servicio (QoS) y mayor flexibilidad en los modos de transmisión. Sin embargo, su despliegue exige hardware especializado, infraestructura de red avanzada y una gestión compleja del espectro, factores que dificultan su adopción en entornos académicos y de simulación.

\subsection{Plataformas abiertas sobre ITS-G5}
    \subsubsection{OpenC2X}
    OpenC2X \cite{OpenC2X} \cite{Demo_OpenC2X} implementa, además de sus módulos principales y el control de congestión (DCC), el estándar IEEE 802.11p en el kernel de Linux y aplica la priorización de tráfico mediante EDCA para diferenciar los mensajes CAM y DENM.

    En sus conclusiones, los autores describen el sistema como “una plataforma experimental y de prototipado de código abierto compatible con el estándar ETSI ITS G5”, señalando que complementa implementaciones anteriores que carecían de funcionalidades clave como la gestión de DCC o LDM.

    Asimismo, destacan que su carácter abierto y arquitectura extensible permiten realizar pruebas reales y añadir nuevos módulos, tales como GeoNetworking o mecanismos de seguridad, desarrollados por la comunidad investigadora.

    \paragraph{Arquitectura y módulos de OpenC2X.} 

    Además de los módulos principales de comunicación vehicular, OpenC2X incluye una arquitectura modular inspirada en el estándar ETSI ITS-G5, donde cada servicio se ejecuta como un proceso independiente comunicado mediante ZeroMQ. Esta organización permite extender y reemplazar fácilmente componentes, así como realizar pruebas experimentales con configuraciones personalizadas.

    En la Figura~\ref{fig:openc2x-architecture} se muestra la arquitectura lógica de OpenC2X, basada en capas funcionales que incluyen servicios de percepción, gestión de posición, generación de mensajes y control de congestión.

    \begin{figure}[H]
        \centering
        \includegraphics[width=0.6\textwidth]{images/openc2x_architecture.png}
        \caption{Arquitectura modular de OpenC2X \cite{tfm-g1791}.}
        \label{fig:openc2x-architecture}
    \end{figure}

    Los módulos más relevantes para los sistemas V2X implementados son:

    \begin{itemize}
        \item \textbf{CAM (Cooperative Awareness Message):} Responsable de generar los mensajes CAM que describen posición, velocidad y estado del vehículo.
        \item \textbf{DENM (Decentralized Environmental Notification Message):} Encargado de gestionar eventos y situaciones de riesgo en el entorno.
        \item \textbf{GPS:} Proporciona la información de posición que utilizan los módulos CAM y DENM para completar sus campos.
        \item \textbf{LDM (Local Dynamic Map):} Base de datos local que almacena objetos, eventos y mensajes recibidos.
        \item \textbf{DCC (Decentralized Congestion Control):} Controla el uso del canal de comunicaciones para evitar saturación y regular la tasa de transmisión.
    \end{itemize}

    Trabajos previos, como el presentado en \cite{tfm-g1791}, emplean esta arquitectura modular para desarrollar funcionalidades de V2X, demostrando que OpenC2X es una plataforma madura y flexible para investigación
    en comunicaciones vehiculares.

    \subsubsection{AutoC2X}
    La plataforma AutoC2X \cite{AutoC2X} amplía las capacidades de OpenC2X al integrar la percepción cooperativa entre los vehículos autónomos. Esta solución combina el software de conducción autónoma Autoware con la plataforma OpenC2X, permitiendo que los vehículos intercambien información sensorial sobre objetos detectados en su entorno.

    El sistema se diseñó para que un vehículo “host” ejecute Autoware (encargado de la detección y localización mediante sensores LiDAR y cámaras), mientras que un “router” ejecuta OpenC2X, encargado de la comunicación ITS-G5 y del intercambio de mensajes V2X entre vehículos y unidades de carretera (RSU).

    El proceso de percepción cooperativa se basa en el envío de la información sensorial de los objetos detectados por Autoware hacia OpenC2X a través de TCP/IP. Esta información se transforma en mensajes proxy-CAM, que permiten representar objetos que no poseen capacidad de transmisión V2X propia.

    Cuando un vehículo recibe estos mensajes, el módulo OpenC2X decodifica la información, la almacena en el Local Dynamic Map (LDM) y la reenvía a Autoware para su visualización.

\subsection{SUMO (Simulation of Urban Mobility)}
SUMO es un paquete de simulación de tráfico continuo, microscópico, altamente portátil y de código abierto, diseñado para gestionar grandes redes. Permite la simulación intermodal, incluyendo peatones, e incluye un amplio conjunto de herramientas para la creación de escenario \cite{SUMO}.

Las caracteristicas más destacadas de SUMO son \cite{Sumo_article}:
\begin{itemize}
    \item Simulación microscópica: los vehículos, los peatones y el transporte público se modelan de forma explícita. 
    \item Simulación de tráfico multimodal, por ejemplo, vehículos, transporte público y peatones.
    \item Los horarios de los semáforos pueden importarse o generarse automáticamente con SUMO.
    \item Sin limitaciones artificiales en cuanto al tamaño de la red y el número de vehículos simulados.
    \item Formatos de importación compatibles: OpenStreetMap, VISUM, VISSIM, NavTeq.
    \item SUMO está implementado en C ++ y utiliza una biblioteca portátil exclusiva.
\end{itemize}

SUMO es ampliamente utilizado por la comunidad V2X para generar trazas realistas de vehículos y evaluar aplicaciones en un bucle en línea con un simulador de red.

\subsection{Desempeño y fiabilidad en comunicaciones V2V}
La comunicación vehículo a vehículo (V2V) es un componente esencial de las aplicaciones cooperativas, ya que permite el intercambio rápido y fiable de información entre vehículos en movimiento, como el platooning, donde los vehículos deben coordinarse para responder instantáneamente.

El artículo \cite{platooning}, propone una plataforma redundante basada en OpenC2X, capaz de operar simultáneamente en las bandas de 5.9GHz y 700MHz, con el objetivo de mejorar la cobertura y la estabilidad de la conexión.

Para aplicaciones como el platooning, puede ser necesaria una latencia de capa de aplicación de extremo a extremo de aproximadamente 40 ms o menos, lo que se consigue mediante DSRC. Este valor constituye una referencia práctica aceptada para sistemas V2V de baja latencia.
Sin embargo, el rendimiento de las comunicaciones DSRC se degrada con la distancia y en entornos urbanos o suburbanos, donde los obstáculos y la difracción reducen la relación señal/ruido. Por este motivo, el trabajo introduce una banda suplementaria a 700 MHz, más robusta ante obstrucciones, con el objetivo de mantener la conectividad cuando la comunicación a 5.9 GHz se interrumpe.

Se realizaron pruebas con hardware embebido y OpenC2X configurado en modo OCB, los cuales mostraron que el canal a 700MHz mantiene una mayor robustez ante pérdidas de señal. Concretamente, para distancias superiores a 130m:
\begin{itemize}
    \item La relación señal-ruido (SNR) de 5.9GHz es bastante baja.
    \item La SNR de 700 MHz es aproximadamente 15 dB más alta.
    \item La latencia es inferior a 30 ms para ambas señales de comunicación.
\end{itemize}
Esta mejora confirma que el uso de una segunda banda de 700MHz permite compensar la degradación de la comunicación en entornos con obstáculos o interferencias.

% \subsection{Seguridad V2X: detección de intrusos}
% Los sistemas de detección de intrusiones (IDS) tradicionales \cite{intrusion_detection} basados en el aprendizaje automático se basan en el procesamiento centralizado de datos, lo que plantea retos como riesgos para la privacidad, limitaciones de ancho de banda y una elevada sobrecarga computacional. Estas limitaciones dificultan la escalabilidad y la adaptabilidad en tiempo real en entornos V2X altamente dinámicos. Los IDS tradicionales, que se basan en arquitecturas centralizadas, se enfrentan a varias limitaciones críticas cuando se aplican a entornos V2X.

% Para superar estas restricciones, se propone la \textbf{adopción de aprendizaje federado (Federated Learning, FL)}, que utiliza un enfoque descentralizado de aprendizaje automático en el que los vehículos y las RSU colaboran para entrenar un modelo global de detección de intrusiones sin compartir datos sin procesar, y \textbf{computación en el borde (Edge AI)}, que implementa modelos de IA ligeros directamente en los vehículos y dispositivos periféricos, lo que permite la detección de amenazas en tiempo real sin depender del procesamiento basado en la nube.

% Aunque la seguridad no constituye el foco principal de este trabajo, los avances en detección distribuida mediante Edge AI y aprendizaje federado representan un área de mejora futura para las plataformas abiertas como OpenC2X.