\section{Diseño y Arquitectura del sistema}
\subsection{Vista general}
En esta sección se presenta la arquitectura general del sistema, mostrando los principales módulos involucrados y el flujo de información entre ellos.
\begin{figure}[H]
    \centering
    \includegraphics[width=0.7\textwidth]{images/Diagrama de bloques.PNG}
    \caption{Arquitectura del sistema SUMO $\leftrightarrow$ OpenC2X (escenario V2V)}
    \label{fig:diagrama_de_bloques}
\end{figure}

\subsection{Bloques y funciones (descripción breve)}

\begin{enumerate}[label=\arabic*.]
    \item \textit{SUMO (escenario de tráfico)}
    \begin{itemize}
        \item Simula la movilidad de los vehículos (trayectorias, velocidades, interacciones).
        \item Ficheros: red (.net.xml), rutas (.rou.xml) y configuración (.sumocfg).
        \item Expone los estados de cada vehículo a través de \textit{TraCI}.
    \end{itemize}

    \item \textit{Módulo SumoInterface (bridge TraCI $\leftrightarrow$ OpenC2X)}
    \begin{itemize}
        \item El módulo SumoInterface implementa un cliente \textit{TraCI} basado en la librería oficial libtraci (SUMO 1.23.1), estableciendo una conexión TCP con SUMO en el puerto configurado (por defecto, 9999).
        \item En cada tick, este cliente consulta las variables de movilidad (posición, velocidad, orientación) y publica los estados actualizados a los nodos \textit{OpenC2X}.
    \end{itemize}

    \item \textit{Nodos OpenC2X (Vehículo A, B, …)}
    \begin{itemize}
        \item Procesos independientes que representan "vehículos conectados".
        \item \textit{CAM}: generación y recepción de mensajes V2V.
        \item \textit{GPS}: actualiza la posición usando los datos recibidos desde SUMO.
        \item \textit{DENM}: módulo encargado de generar mensajes de alerta a partir de \textit{triggers} internos o externos.
        \item Otros (DCC, LDM)
    \end{itemize}

    \item \textit{Módulo de registro}
    \begin{itemize}
        \item Recoge logs de los nodos (timestamps, IDs, payloads, pérdidas).
        \item Genera métricas básicas.
        \item Exporta los resultados.
    \end{itemize}
\end{enumerate}

\subsection{Flujo de datos}
\begin{enumerate}[label=\arabic*.]
    \item \textit{SUMO} avanza el tiempo y actualiza los estados vehiculares.
    \item \textit{SumoInterface (TraCI)} lee el estado de cada vehículo y lo distribuye a los nodos \textit{OpenC2X} correspondientes.
    \item Cada nodo \textit{OpenC2X}:
    \begin{itemize}
        \item Actualiza el \textit{GPS}.
        \item Genera y recibe mensajes \textit{V2X} mediante \textit{CAM}.
    \end{itemize}
    \item Los mensajes \textit{V2V} se intercambian entre nodos mediante \textit{sockets} internos.
    \item El módulo \textit{DENM} puede generar alertas a partir de \textit{triggers} externos recibidos (flujo V2I), activados cuando \textit{SumoInterface} detecta un evento en la infraestructura.
    \item El módulo de registro revisa los logs y genera las métricas básicas.
    \item El usuario revisa las métricas y ajusta los parámetros o escenarios.
\end{enumerate}

\subsection{Entorno físico y red}
\begin{itemize}
    \item \textit{Host con virtualización}
        \begin{itemize}
            \item VirtualBox.
            \item Máquina virtual configurada con \textit{Ubuntu 24.04 LTS}, \textit{10 GB de RAM}, \textit{2 núcleos asignados} y \textit{50 GB de almacenamiento en disco}.
        \end{itemize}
    \item \textit{Red}: comunicación intra-host basada en \textit{sockets}; no se requiere de una interfaz inalámbrica real.
    \item \textit{Escalabilidad}: añadir más vehículos implica más procesos/nodos \textit{OpenC2X} y más suscripciones \textit{TraCI}.
\end{itemize}

\subsection{Interacción con el Usuario}
La ejecución se realiza mediante scricps de arranque, con los parámetros en los ficheros de configuración. Los resultados se analizan revisando los logs generados por \textit{OpenC2X}.